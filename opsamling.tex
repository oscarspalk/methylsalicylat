\section*{Opsamling}
\subsection*{Sammenligning}
Kigger vi på værdierne fra del 1 og 2 og databogen, ses i nedenstående tabel,
har vi helt sikkert ramt tæt på.
Værdierne for $pK_s$ ligger alle sammen på omkring 4.

\begin{table}[h]
    \centering
    \begin{tabular}{|c|c|c|c|}
    \hline

    Del & 1 & 2 & Databog \\ \hline
    $K_s/\textsc{m}$ & $6{,}15 \cdot 10^{-5}$ & $2{,}51 \cdot 10^{-5}$ & $6{,}76 \cdot 10^{-5}$ \\ \hline
    $pK_s$ & 4{,}21 & 4{,}6 & 4{,}17 \\ \hline
    \end{tabular}
\end{table}

For $K_s$ værdierne er det lidt anderledes, de afviger nemlig en smule mere.
Umiddelbart ud fra værdierne er det sikkert at sige, at del 1 evt.~er en bedre metode
til at bestemme $K_s$ og $pK_s$.

\subsection*{Fejlkilder}
Der er ingen tvivl om, at vi har lavet en fejl, da vi fortyndede opløsningerne i Del 1.
For delforsøg 2 er der også en måleusikkerhed, da vi aflæser manuelt fra titrerkurven.
Automatisk aflæsning ville nok føre til resultater, der ville afvige mindre fra tabel-værdierne.

\subsection*{Konklusion}
Vha.~pH-målinger kan det sluttes, at ascorbinsyre, hvis man regner den som en monohydronsyre,
har $pK_s \approx 4{,}21$ og tilsvarende $K_s \approx 6{,}15 \cdot 10^{-5}\ \textsc{m}$.