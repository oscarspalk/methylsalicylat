\subsection*{Teori}\label{Teori}
Methylsalicylat kan dannes ved kondensering af salicylsyre
(2-hydroxy-benzoesyre) og methanol, som på reaktionsskema \ref{fig:rek1}.

\begin{figure}[h]
  \centering
  \ce{C6H6OHCOOH (aq) + CH3OH (aq) <=> C6H6OHCOOCH3 (aq) + H2O (l)}
  \caption{Kondensation af methylsalicylat}
  \label{fig:rek1}
\end{figure}

Reaktionsskema~\ref{fig:rek1} kan også ses på strukturformel på
reaktionsskema~\ref{fig:rek2}

\begin{figure}[h]
  \centering
  \scalebox{.9}{
  \schemestart
  \chemfig{**6((- (=_[::-60]O)-[::60]OH)- (-OH)-----)}
  \+
  \chemfig{C (-[::0]H) (-[::90]OH) (-[::180]H) (-[::270]H)}
  \arrow{<=>}
  \chemfig{
    **6((- (-[::-60]O-)=_[::60]O)- (-OH)-----)
  }
  \+
  \chemfig{O (-[::60]H) (-[::-60]H)}
  \schemestop
  }
  \caption{Kondensation af methylsalicylat}
  \label{fig:rek2}
\end{figure}

Eftersom at kondenseringen af methylsalicylat er en
ligevægts reaktion, vil viden om Le Chateliers princip være nyttigt.
Hans princip siger, at et ydre indgreb i et system vil medføre
en forskydning. I kondenseringen fjerner vi $\ce{H2O}$ vha.~bla.~svovlsyre,
og dermed forskydes reaktionen mod højre, og der vil blive
produceret mere methylsalicylat.

Derudover anvendes der også IR-spektroskopi. Det vil sige,
at en maskine sender infrarød stråling mod et stof,
hvorefter maskinen måler, hvilke bølgelængder, der bliver
absorberet. Forskellige bindinger og funktionelle grupper har
nemlig nogle specifikke bølgelængder, som de absorberer,
pga.~vibrationer i molekyler og atomer. På den måde bestemmes
kompositionen af vores synteseprodukt.