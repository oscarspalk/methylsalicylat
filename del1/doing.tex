\subsection*{Udførelse}
Først tilsatte afmålte vi $9\ \unit{\gram}$ salicylsyre
med $0{,}01\ \unit{\gram}$'s nøjagtighed til en rundbundet kolbe,
hvorefter vi tilsatte $25\ \ml$ methanol.
$25\ \ml$ methanol er egentlig langt mere, end vi har brug for, men ved at tilsætte methanol 
i overskud mindsker vi reaktionstiden, da vi forøger koncentrationen og som Le Chatelier siger, forskyder vi reaktionen mod højre (mere methylsalicylat).

Så tilsatte vi $25\ \ml$ koncentreret svovlsyre
og puttede nogle få pimpsten i kolben,
som vi anbragte i vores refluksopstilling og lod det simre i $\frac{1}{2}$ time.
Udover varmes effekt på reaktionstiden, da det er en endoterm mod højre reaktion, så vil varmen resultere i noget vand, som 
vil gå på gasform. Det vil betyde, at ligevægten skubbes 
endnu længere mod højre.
Vi anvendte en refluksopstilling her, da selvom methylsalicylat har et væsentligt højere kogepunkt,
så vil lidt af stoffet gå på gasform,
og med en refluksopstilling, så vil det ende i 
vores rundbundede kolbe igen.

Da opløsningen var taget af refluksopstillingen og var afkølet,
så overførte vi opløsningen til en skilletragt og tilsatte
$25\ \ml$ dichlormethan sammen med $20\ \ml$ demivand.
Dichlormethan er et perfekt stof til vores syntese,
da det er et upolært stof, som vi kan opløse methylsalicylaten
i. Og så er det meget nemmere at fjerne end vand,
da dichlormethans kogepunkt er $39\ \unit{\celsius}$.

Så rystede vi forsigtigt og udluftede undervejs,
så den organiske fase var adskilt og klar til at blive tappet.
Det gjorde vi selvfølgelig,
da vi havde fjernet proppen.
Bagefter tilsatte vi igen $25\ \ml$ dichlormethan
og gentog de andre skridt for at adskille
den organiske fase.

Dernæst tappede vi vand fasen til et affaldsglas,
overførte den organiske fase til skilletragten igen,
men vi skyllede med vand i stedet for dichlormethan,
og så tappede vi igen.

Og så overførte vi den organiske fase til skilletragten igen,
hvor vi puttede $25\ \ml$ $5\%$ natriumcarbonat i,
og så tapper vi den organiske fase,
og skiller os af med vand fasen. Dette gentages 1 gang yderligere.
Skylle delen af trinnet er for at adskille vores faser,
og natriumcarbonaten skal fjerne eventuelle rester af svovlsyre.

I bægerglasset med den organiske fase
tilsatte vi så lidt magnesiumsulfat,
som hjalp med at suge resterne af vandet.
Så blev resten af opløsningen kørt gennem et filter
og ned i en rundbundet kolbe,
som vi satte ind i vores destillationsopstilling.

Vi vejede vores rundbundede kolbe før, vi begyndte at destillere,
på den måde vidste vi ca., hvornår vi var færdige
med at destillere på vores opløsning.
