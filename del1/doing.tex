\subsection*{Udførelse}
Først hentede vi en frisk fremstillet $0{,}1\ \textsc{m}$ ascorbinsyre opløsning,
og så fremstilte vi en $0{,}01\ \textsc{m}$ ascorbinsyre opløsning ved at fortynde
$1\ \unit{\milli\liter}$ af $0{,}1\ \textsc{m}$ ascorbinsyre med $9\ \unit{\milli\liter}$ demineraliseret vand.
Bagefter gentog vi fortyndingen,
hvor vi tog $1\ \unit{\milli\liter}$ af vores $0{,}01\ \textsc{m}$ ascorbinsyre,
som vi så fortyndede med $9\ \unit{\milli\liter}$ demineraliseret vand,
så resultatet var $0{,}001 \ \textsc{m}$ ascorbinsyre.

Så skulle vi kalibrere pH-elektroden vha.~pufferopløsningerne, som vi brugte til at måle pH-værdierne for vores tre opløsninger.

\subsection*{Målinger}
Alle pH-målinger er indskrevet i nedenstående tabel.

\begin{table}[h]
    \centering
    \begin{tabular}{|c|c|c|c|}
    \hline
    $c_s(\ce{C6H8O6})/\textsc{m}$ & 0{,}1 & 0{,}01 & 0{,}001 \\ \hline
    pH & 2{,}64 & 3{,}1 & 3{,}45 \\ \hline
    \end{tabular}
\end{table}

\subsection*{Databehandling}
For at beregne $K_s$ og $pK_s$ vil vi opskrive reaktionsbrøker for hver af vores opløsninger.
Dette kræver, at vi kender koncentrationerne af reaktanterne og produkterne med undtagelse af vand.

Først beregner vi koncentrationen af \ce{H3O+}, hvilket vi gør ved at vende pH om:

\begin{align*}
    pH &=-\log_{10}[\ce{H3O+}]
    \\
    \Updownarrow
    \\
    [\ce{H3O+}] &=10^{-pH}
\end{align*}

Dernæst finder vi mængden af \ce{C6H7O6-}, som må være lig koncentrationen af \ce{H3O+}, eftersom der bliver dannet lige mange for hver reaktion.

Til sidst må koncentrationen af \ce{C6H8O6} være sin start koncentration ${[\ce{C6H8O6}]}_{start}$,
hvor vi trækker ${[\ce{C6H7O6-}]}$ fra, da det er den mængde, der er blevet omdannet, derfor:

\begin{table}[h]
    \centering
    \begin{tabular}{|c|c|c|c|}
    \hline
    $c_s(\ce{C6H8O6})/\textsc{m}$ & 0{,}1 & 0{,}01 & 0{,}001 \\ \hline
    pH & 2{,}64 & 3{,}1 & 3{,}45 \\ \hline
    $[\ce{H3O+}]/\textsc{m}$ & 0{,}00229 & 0{,}000794 & 0{,}000345 \\ \hline
    $[\ce{C6H7O6-}]/\textsc{m}$ & 0{,}00229 & 0{,}000794 & 0{,}000345 \\ \hline
    $[\ce{C6H8O6}]/\textsc{m}$ & 0{,}097 & 0{,}0092 & 0{,}00065 \\ \hline
    \end{tabular}
\end{table}

Nu kan vi opstille reaktionsbrøken fra teori afsnittet for hver enkel koncentration:

\begin{table}[h]
    \centering
    \begin{tabular}{|c|c|c|c|}
    \hline
    $c_s(\ce{C6H8O6})/\textsc{m}$ & 0{,}1 & 0{,}01 & 0{,}001 \\ \hline
    $K_s/\textsc{m}$ & 0{,}000054 & 0{,}000069 & 0{,}00018 \\ \hline
    \end{tabular}
\end{table}

Vi vælger at se bort fra $K_s$ for $0{,}001\ \textsc{m}$ ascorbinsyre,
da $K_s$ for $0{,}1\ \textsc{m}$ og $0{,}01\ \textsc{m}$ er relativt meget tættere på hinanden.

For at afslutte vores estimering af $K_s$, så tager vi gennemsnittet af de 2 værdier:

\[
    K_{s,gennemsnit}=\frac{0{,}000054\ \textsc{m} +0{,}000069\ \textsc{m}}{2}=0{,}0000615\ \textsc{m}
\]

Nu kan vi forholdsvist let beregne $pKs$ ved at tage $-\log_{10}(K_s)$:

\[
    pK_s=-\log_{10}(0{,}0000615)=4{,}21
\]

Hvilket virker fornuftigt ift.~databogens $pK_s=4{,}17$, hvilket vi vender tilbage til.