\documentclass{article}
% packages % 
\usepackage[a4paper]{geometry}
\usepackage{graphicx}
\usepackage[version=4]{mhchem}
\usepackage{siunitx}
\usepackage{multicol}
\usepackage{wrapfig}
\usepackage[utf8]{inputenc}
\usepackage[danish]{babel}
\usepackage[dvipsnames]{xcolor}
\usepackage{chemfig}
\usepackage[T1]{fontenc}
\usepackage{caption}
\usepackage[hidelinks]{hyperref}

\hypersetup{
    colorlinks=false,
    linkcolor=blue,
    filecolor=magenta,      
    urlcolor=cyan,
    pdftitle={SRO i Kinematik},
    pdfpagemode=FullScreen,
}
\captionsetup[figure]{name={Reaktionsskema}}
\graphicspath{{./images/}}

\renewcommand{\thesection}{\Roman{section}} 
\DeclareMathSymbol{*}{\mathbin}{symbols}{01}

\newcommand{\Title}{Syntese af methylsalicylat}
\newcommand{\Author}{Oscar 2.bx}
\newcommand{\Institution}{Egaa Gymnasium}
\newcommand{\Subtitle}{}
\newcommand{\Date}{\today}
\newcommand{\pColor}{BrickRed}

\begin{document}

\begin{titlepage} % Suppresses displaying the page number on the title page and the subsequent page counts as page 1
	{
        \vspace*{-2cm}
	\raggedleft%
	
	\rule{1pt}{\textheight} % Vertical line
	\hspace{0.05\textwidth} % Whitespace between the vertical line and title page text
	\parbox[b]{0.75\textwidth}{ % Paragraph box for holding the title page text, adjust the width to move the title page left or right on the page
		
		{\huge \bfseries Syntese af methylsalicylat}\\[2\baselineskip] % Title
		{\Large\textsc{Oscar 2.bx}} % Author name, lower case for consistent small caps
		
		\vspace{0.5\textheight} % Whitespace between the title block and the publisher
		
	}
    }

\end{titlepage}

\pagebreak


\pagebreak

\section*{Introduktion}
\subsection*{Formål}
Formålet med eksperimentet er at bestemme syrestyrkekonstanten,
$K_s$, og styrkeeksponenten, $pK_s$, for ascorbinsyre vha.~pH-målinger.
\subsection*{Teori}\label{Teori}
Methylsalicylat kan dannes ved kondensering af salicylsyre
(2-hydroxy-benzoesyre) og methanol, som på reaktionsskema \ref{fig:rek1}.

\begin{figure}[h]
  \centering
  \ce{C6H6OHCOOH (aq) + CH3OH (aq) <=> C6H6OHCOOCH3 (aq) + H2O (l)}
  \caption{Kondensation af methylsalicylat}
  \label{fig:rek1}
\end{figure}

Reaktionsskema~\ref{fig:rek1} kan også ses på strukturformel på
figur~\ref{fig:rek2}

\begin{figure}[h]
  \centering
  \scalebox{.9}{
  \schemestart
  \chemfig{**6((- (=_[::-60]O)-[::60]OH)- (-OH)-----)}
  \+
  \chemfig{C (-[::0]H) (-[::90]OH) (-[::180]H) (-[::270]H)}
  \arrow{<=>}
  \chemfig{
    **6((- (-[::-60]O-)=_[::60]O)- (-OH)-----)
  }
  \+
  \chemfig{O (-[::60]H) (-[::-60]H)}
  \schemestop
  }
  \caption{Kondensation af methylsalicylat}
  \label{fig:rek2}
\end{figure}


\subsection*{Hypotese}
Vi forventer, at de to metoders målinger vil resultere i nogenlunde ens resultater.
\subsection*{Udstyr}
\begin{multicols}{2}
    \begin{itemize}
        \item Salicylsyre
    \end{itemize}    
\end{multicols}



\end{document}
